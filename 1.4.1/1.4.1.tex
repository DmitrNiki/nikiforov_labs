\documentclass[a4paper,14pt]{extarticle}
\usepackage[a4paper,top=1.3cm,bottom=2cm,left=1.5cm,right=1.5cm,marginparwidth=0.75cm]{geometry}
\usepackage{setspace}
\usepackage{cmap}					
\usepackage{mathtext} 				
\usepackage[T2A]{fontenc}			
\usepackage[utf8]{inputenc}			
\usepackage[english,russian]{babel}
\usepackage{multirow}
\usepackage{graphicx}
\usepackage{wrapfig}
\usepackage{tabularx}
\usepackage{float}
\usepackage{longtable}
\usepackage{hyperref}
\hypersetup{colorlinks=true,urlcolor=blue}
\usepackage[rgb]{xcolor}
\usepackage{amsmath,amsfonts,amssymb,amsthm,mathtools} 
\usepackage{icomma} 
\mathtoolsset{showonlyrefs=true}
\usepackage{euscript}
\usepackage{mathrsfs}

\DeclareMathOperator{\sgn}{\mathop{sgn}}
\newcommand*{\hm}[1]{#1\nobreak\discretionary{}
	{\hbox{$\mathsurround=0pt #1$}}{}}

\newcommand{\RomanNumeralCaps}[1]
{\MakeUppercase{\romannumeral #1}}

\usepackage{soulutf8} 
\usepackage{geometry}



\begin{document}
	\begin{center}
		\textit{Федеральное государственное автономное образовательное\\ учреждение высшего образования }
		
		\vspace{0.5ex}
		
		\textbf{«Московский физико-технический институт\\ (национальный исследовательский университет)»}
	\end{center}
	
	\vspace{10ex}
	
	
	\begin{center}
		\vspace{13ex}	
		\textbf{Лабораторная работа №1.4.1}	
		\vspace{1ex}
		
		по курсу общей физики		
		на тему:		
		\textbf{\textit{<<Изучение физического маятника>>}}		
		\vspace{30ex}
		
		\begin{flushright}
			\noindent
			\textit{Работу выполнил:}\\  
			\textit{Никифоров Дмитрий \\(группа Б02-205)}
		\end{flushright}
		\vfill
		Долгопрудный \\ \today
		
		%\setcounter{page}{1}
	\end{center}
	\newpage
	
	\section{Введение}
	
	\textbf{Цель работы:}
	Исследовать зависимость периода колебаний физического маятника от его момента инерции; сравнить зависмость с теоретической. Убедиться в справедливости теоремы Гюйгенса об обратимости точек опоры и центра качания маятника. Определить добротность физического маятника.
	
	\textbf{Оборудование:}
	Физичексий маятник(однородный стальной стержень), математический маятник, механический счетчик колебаний, опорная призма;\\
    \textit{Измерительные приборы:} линейка - $\sigma_l = c = 0,1 \text{см}$, секундомер - $\sigma_t = c = 0,01 \text{с}$.
    
    \section{Теоретическая справка}
    
    Физический маятник - твердое тело, способное совершать колебания в вертикальной плоскости, будучи подвешенным за одну из своих точек в поле тяжести.
    Движение маятника можно представить ввиде вращения его центра масс относительно оси вращения.\\ Уравнения вращательного движения точки: $$M = \frac{dL}{dt}$$, где $L = m\omega r^2$ - момент импульса маятника, а $r$ - расстояние от оси вращения до центра масс;\\
    Момент импульса можно выразить через момент инерции маятника: $L = I\omega$. Тогда перезапишем наше уравнение так:
    $$M = I\frac{d\omega}{dt}$$
    В данной работе роль физического маятника играет однородный стержень постоянного сечения, длины - $l$ . Его момент инерции относительно центра масс: $I_\text{ц.м.} = \frac{ml^2}{12}$. Нас же интересует момент инерции относительно оси вращения (обозначим расстояние до нее от центра масс - $a$). По теореме Гюйгенса-Штейнера он равен: $$I  = \frac{ml^2}{12} + ma^2$$
    Расписав моменты сил, действующих на маятник, относительно точки вращения, получим(считая колебания малыми):
	$$M = -mga\sin \varphi \approx -mga\varphi$$ 
	Заметим - $\omega = \dot{\varphi}$; Таким образом получаем дифференциальное уравнение:
	$$-mga\varphi = I\ddot{\varphi} \Leftrightarrow \ddot{\varphi} = -\frac{mga}{I}\varphi$$
	Данное уравнение описывает гармонические колебания с циклической частотой: $$\omega = \sqrt{\frac{mga}{I}}$$
	Таким образом период колебаний физического маятника определяется по формуле:
	$$T = 2\pi \sqrt{\frac{\frac{l^2}{12} + a^2}{ga}}$$
	Сравним ее с формулой периода для математического маятника: $T = 2\pi \sqrt{\frac{l\text{пр}}{g}}$.
	Приравняв периоды, получим приведенную длину физического маятника:
	$$l\text{пр} = \frac{l^2}{12a} + a$$
	Рассмотрим затухание колебаний; Очевидно амплитуда колебаний монотонно убывает с течением времени: A(t) - убывающая функция. Относительную убыль амплитуды (за малое $dt$)принято называть декрементом затухания: $\gamma = \frac{|{dA}|}{A}$. В большинстве случаев ее можно считать постоянной во времени: $\gamma(t) = \frac{-{dA}}{A} = const$. Проинтегруя это уравнение, получим зависимость амплитуды от времени:$$A(t) = A_0 e^{-\gamma t}$$
	Величину $\frac{1}{\gamma}$ - будем называть характерным временем затухания ($\tau_\text{зат}$), за которое амплитуда уменьшается в $e$ раз.\\    
	Добротность колебательной системы($Q$) - безразмерная характеристика затухания. Будем рассчитывать ее по формуле: $$Q = \pi \frac{\tau_\text{зат}}{T} $$. Для удобства в нашем эксперименте будем рассматривать затухание в 2 раза, т.е. полученное время($\tau$) нужно будет поделить на $\ln{2}$ для определения характерного времени. Таким образом добротность будем рассчитывать по формуле: $$Q = \pi \frac{\tau}{\ln{2} T} = \pi \frac{n}{\ln{2}}$$, где $n$ - количество полных колебаний, совершенный маятником за время $\tau$.
	\section{Описание установки}
	Экспериментальная установка представляет из себя стальной стержень, играющий роль физического маятника, с закрепленной на нем металлической призмой с острым ребром. Эту систему устанавливают в равновесие так, что острое ребро призмы задает ось вращения стержня, при его отклонении.  	   
\end{document}   
	 