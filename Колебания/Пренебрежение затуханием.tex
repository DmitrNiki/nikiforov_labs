
\documentclass[a4paper,14pt]{extarticle}
\usepackage[T2A]{fontenc}
\usepackage[utf8]{inputenc}
\usepackage[english,russian]{babel}
\usepackage{amsmath,amsfonts,amsthm, mathtools}
\usepackage{amssymb}
\usepackage{icomma}
\usepackage{graphicx}
\usepackage{wrapfig}
\RequirePackage{longtable}
\newcommand*{\hm}[1]{#1\nobreak\discretionary{}
	{\hbox{$\mathsurround=0pt #1$}}{}}

\usepackage{soulutf8} 
\usepackage{geometry}
\geometry{top=20mm}
\geometry{bottom=20mm}
\geometry{left=20mm}
\geometry{right=20mm}
\begin{document}
\begin{flushleft}
Уравнение затухающих колебаний:
$$x=Ae^{-\delta t}{\cos  \left(\omega t+{\varphi }_0\right)\ }\left(4\right)$$
, где $\omega =\sqrt{{\omega }^2_0-{\delta }^2}$ — частота затухающих колебаний, $Ae^{-\delta t}$ — амплитуда затухающих колебаний. ${\varphi }_0$— постоянная величина, которая зависит от выбора начала отсчета времени.

Коэффициент затухания можно определить как величину обратную времени ($\tau$) за которое амплитуд (A) уменьшается в e раз:

$$\delta =\frac{1}{\tau }\left(5\right)$$

, где $\tau$ — время релаксации. То есть можно записать:

$$\frac{A_1}{A_2}=e^{\delta t} \Leftrightarrow \delta = \frac{\ln(\frac{A_1}{A_2})}{t} \left(6\right)$$

Период затухающих колебаний равен:

$$T=\frac{2\pi }{\sqrt{{\omega }^2_0-{\delta }^2}}\left(7\right)$$
при несущественном сопротивлении среды, если выполняется неравенство: ${\delta }^2\ll {\omega }^2_0$ период колебаний можно вычислять при помощи формулы:

$$T'=\frac{2\pi }{{\omega }_0}\left(8\right)$$
рассмотрим $\delta^2$, при условии $\left(*\right) -$ за 10 периодов колебаний амплитуда меняется меньше чем в два раза($\frac{A_1}{A_2} < 2$):
$$\ln{\frac{A_1}{A_2}} < 1,\;\;\;\; \omega_0 = \frac{2\pi}{T'}$$ 
$$\frac{T'}{T} = \frac{\sqrt{\omega_0^2 - \delta^2}}{\omega_0} < 1 \Rightarrow T' < T$$
$$\text{, из $\left(6\right)$:}\;\; \delta^2 = \frac{(\ln{\frac{A_1}{A_2}})^2}{100T^2} \ll \frac{1}{T'^2} < \frac{4\pi^2}{T'^2} = \omega_0^2$$
, таким образом, если выполняется $\left(*\right)$ можно считать колебания незатухающими. Погрешность такого пренебрежения: 
$$\sigma_T^{\text{сист}} < \frac{2\pi}{\omega_0}(\frac{1}{\sqrt{0,99}} - 1),\; \text{т.е.} \;\; \varepsilon_T^{\text{сист}} < 0,5\% $$

\end{flushleft}
\end{document}
